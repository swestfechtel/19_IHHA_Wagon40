% LaTeX-Vorlage zur Erstellung von Abschlussarbeiten an der FH Aachen
% Author: Sven Hinz
% Aenderung für FB 5: Ingo Elsen

\documentclass[12pt,a4paper]{book}
% Paket für Umlaute:
\usepackage[utf8]{inputenc}       % Cross Platform
%\usepackage[ansinew]{inputenc}   % Windows
%\usepackage[latin1]{inputenc}    % Linux
%\usepackage[applemac]{inputenc}  % Mac

%\usepackage[ngerman]{babel}       % Sprache: deutsch
\usepackage[utf8]{inputenc} % For degree symbol in tex source
\usepackage{amsmath}
\usepackage{amsfonts}
\usepackage{amssymb}
\usepackage{makeidx}
\usepackage{graphicx}
\usepackage{epstopdf}
%\usepackage{kpfonts}
\usepackage{textcomp}
\usepackage[left=2cm,right=2cm,top=2.5cm,bottom=2.5cm]{geometry}
%\usepackage[plainheadsepline,headsepline]{scrpage2}
\usepackage{color}
\usepackage{setspace}
%\usepackage[numbers,square]{natbib} % only required for unsrtd bib style
\usepackage{longtable}
\usepackage{listings}
\usepackage{rotating}
\usepackage{pdfpages}
\usepackage{caption}
\usepackage{subcaption}
\usepackage{float}
\restylefloat{table}
\parindent 0pt
\usepackage{booktabs}
\usepackage[export]{adjustbox}
\usepackage{titlesec}
\usepackage{mathtools}
\usepackage{ulem}
\usepackage{csquotes}
\usepackage{pythonhighlight}
%\usepackage{hyperref}
%\usepackage[cache=false]{minted}

%\titleformat{\chapter}[display]
%{\normalfont\bfseries}{}{0pt}{\Large}

\newcommand{\TODO}[1]{\textbf{\textcolor{red}{$<$TODO$>$ #1 $</>$}}}
\newcommand{\NOTE}[1]{\textcolor{cyan}{NOTE: #1}}

\DeclarePairedDelimiter\Floor\lfloor\rfloor
\DeclarePairedDelimiter\Ceil\lceil\rceil


% Schriftart
%\usepackage{courier}
%\usepackage{helvet}
%\usepackage{times}
%\renewcommand{\familydefault}{\sfdefault}
%\renewcommand{\familydefault}{\rmdefault}
%\setkomafont{chapter}{\sffamily \large}
%\setkomafont{section}{\sffamily \normalsize}
%\setkomafont{subsection}{\sffamily \normalsize}
%\setkomafont{subsubsection}{\sffamily \normalsize}
%\addtokomafont{caption}{\sffamily \small}

\setlength{\parindent}{0.5cm}
% Abstand zwischen Kopfzeile und Kapitelüberschrift
%\renewcommand*{\chapterheadstartvskip}{\vspace*{-0.75\baselineskip}}

% Einstellungen der Kopf- und Fußzeile
%\pagestyle{scrheadings}
%\ihead[\sffamily \bfseries \upshape \headmark]{\sffamily \bfseries \upshape %\headmark}
%\chead[]{}
%\ohead[]{}
%\ifoot[]{}
%\cfoot[]{}
%\ofoot[\sffamily \pagemark]{\sffamily \pagemark}
%\automark[]{chapter}
%\renewcommand*{\chapterheadendvskip}{\vspace*{1\baselineskip}}

% Formeln
\usepackage{fleqn} % linksbündig
\setlength{\mathindent}{1.5cm} % Einrücktiefe

% Tabellen
\usepackage{multirow} % mehrzeiliger Text in einer Spalte
\renewcommand{\arraystretch}{2} % Zeilenabstand vergrößern
\setlength{\doublerulesep}{0.1mm} % Abstand der Doppellinien verkleinern
\usepackage{tabu}
\newcolumntype{C}{>{\centering\arraybackslash$}p{3cm}<{$}}

% Quellcode / Kommandozeileneingabe
\definecolor{dkgreen}{rgb}{0,0.6,0}
\definecolor{gray}{rgb}{0.5,0.5,0.5}
\definecolor{mauve}{rgb}{0.58,0,0.82}
\definecolor{ocker}{rgb}{0.92,0.68,0.2}

\newcommand\matlabstyle{\lstset{
  language=matlab,
  aboveskip=10mm,
  belowskip=10mm,
  basicstyle=\small\ttfamily,
  frame=tb,
  columns=fullflexible,
  commentstyle=\color{gray},
  stringstyle=\color{dkgreen},
  keywordstyle=\color{mauve},
  numberstyle=\color{ocker},
  breaklines=true,
  tabsize=3
}}

\lstnewenvironment{matlab}[1][]
{
	\matlabstyle
	\lstset{#1}
}
{}

%\newcommand\pythonstyle{\lstset{
%		language=Python,
%		basicstyle=\ttm,
%		otherkeywords={self},             % Add keywords here
%		keywordstyle=\ttb\color{deepblue},
%		emph={MyClass,__init__},          % Custom highlighting
%		emphstyle=\ttb\color{deepred},    % Custom highlighting style
%		stringstyle=\color{deepgreen},
%		frame=tb,                         % Any extra options here
%		showstringspaces=false            % 
%}}
%
%
%% Python environment
%\lstnewenvironment{python}[1][]
%{
%	\pythonstyle
%	\lstset{#1}
%}
%{}


% Inhalt
%\renewcaptionname{ngerman}{\contentsname}{Inhalt} % Umbenennung in Inhalt

% Quellenverzeichnis
%\renewcaptionname{ngerman}{\bibname}{Quellenverzeichnis} % Umbenennung in Quellenverzeichnis

%\usepackage[
%  tocindentmanual,
%  tocflat,
%  tocbreaksstrict,
%  toctextentriesleft,
%]{tocstyle}

% Abkürzungsverzeichnis
%\usepackage[intoc]{nomencl}
%\let\abbrev\nomenclature
%\renewcommand{\nomname}{Abkürzungsverzeichnis}
%\setlength{\nomlabelwidth}{.25\hsize}
%\renewcommand{\nomlabel}[1]{#1 \dotfill}
%\setlength{\nomitemsep}{-\parsep}
%\makenomenclature

\usepackage[]{acronym}


\author{Vorname Nachname} % --> Eigenen Namen einfügen


\begin{document}
\setstretch{1.1}
\addtocontents{toc}{\linespread{1}}

% Einbinden der Textinhalte mit '\include{...}'
% Die Dateien mit den Textinhalten befinden sich im Ordner 'doc'

\include{./doc/titelblatt}

\clearpage
\chapter*{Erklärung}\label{erklaerung}
\markboth{Erklärung}{Erklärung}
Ich versichere hiermit, dass ich die vorliegende Arbeit selbstständig verfasst und keine anderen als die im Literaturverzeichnis angegebenen Quellen benutzt habe.

\bigskip

\noindent
Stellen, die wörtlich oder sinngemäß aus veröffentlichten oder noch nicht veröffentlichten Quellen entnommen sind, sind als solche kenntlich gemacht.

\bigskip

\noindent
Die Zeichnungen oder Abbildungen in dieser Arbeit sind von mir selbst erstellt worden oder mit einem entsprechenden Quellennachweis versehen.

\bigskip

\noindent
Diese Arbeit ist in gleicher oder ähnlicher Form noch bei keiner anderen Prüfungsbehörde eingereicht worden.

\vspace{1cm}
\noindent
Aachen, \today %Monat Jahr

\vspace{7cm}
%\section*{Geheimhaltung}\label{geheimhaltung}

%{\large\textbf{Geheimhaltung}}\\


%Die vorliegende Arbeit unterliegt bis ... der Geheimhaltung. Sie darf vorher weder vollständig noch auszugsweise ohne schriftliche Zustimmung des Autors, des betreuendes Referenten bzw. der Firma ... vervielfältigt, veröffentlicht oder Dritten zugänglich gemacht werden.

\include{./doc/danksagung}

% Inhaltsverzeichnis
\clearpage
\makeatletter
\renewcommand*{\@dotsep}{1} % Punktabstand einstellen
\makeatother
\tableofcontents

% Das erste Kapitel soll auf einer ungeraden Seite beginnen.
\cleardoublepage
\setstretch{1.1}

% Nicht benötigte Kapitel können auskommentiert werden
% Für zusätzliche Kapitel müssen weitere Dateien im Ordner 'doc' angelegt werden

%\include{./doc/101kap1} % Einleitung
%\clearpage
\chapter{\textbf{Grundlagen}}\label{grundlagen}
%\addtocontents{toc}{\vspace{0.8cm}}

\section{Unterkapitel}\label{unterkapitel}
\addtocontents{toc}{\vspace{0.8cm}}

Wir sehen im Folgenden die Formel für die Faltung von Wahrscheinlichkeitsdichtefunktionen, als Gleichungsarray:

\begin{align}
(p_i * p_j)(n) & =  \sum_{k \in \mathbb{D}} p_i(k) \cdot p_j(n - k) \\
p_{total} & =  p_0 \ast p_1 \ast \ldots p_{n-1}; \forall n
\end{align}

% Formel
Hier ist nur eine einfache Formel mit der \texttt{equation}-Umgebung für die Minkowski Metrik:
\begin{equation}\label{Minkowski}
D\left(X,Y\right)=\left(\sum_{i=1}^n |x_i-y_i|^p\right)^{1/p}\\
\end{equation}

Wie in Gleichung \ref{Minkowski} zu erkennen ist, ergibt sich die L2-Norm (Euklidische Distanz), wenn man den Exponenten $p = 2$ wählt.

Support Vector Machines \cite{Haykin99} nutzen die Euklidische Distanz (oder äquivalent) das Skalarprodukt.
%% Zwei Abbildungen, die zusammen gehören

%\begin{figure}
%        \centering
%        \begin{minipage}[c]{0.45\textwidth}
%                \includegraphics[height=6.5cm]{pic/dateiname1.png}
%        \end{minipage}
%        \begin{minipage}[c]{0.45\textwidth}
%                \includegraphics[height=6.5cm]{pic/dateiname2.png}
%        \end{minipage}
%        \caption{Zwei Abbildungen}\label{fig:zwei_abb}
%\end{figure}

%\clearpage
\chapter{\textbf{Kapitel 3}}\label{kap3}
\addtocontents{toc}{\vspace{0.8cm}}

\begin{table}[htb]
\caption{Messergebnisse}
\label{tab:messung}
\centering
\begin{tabu}{l|[2pt]C|C|C}
Stellung & \frac{T_U}{^\circ C}  & \frac{T_c}{^\circ C} & \frac{\Delta T}{^\circ C}  \\
\tabucline[2pt]{-}
senkrecht (0°) & 27,3 & 69,8 & 42,5\\
\tabucline[0.5pt]{-}
waagerecht (90°) & 26,6 & 70,6 & 44,0\\
\end{tabu}
\end{table}

\begin{table}[h]
\centering
\caption{Smartphone Sensordaten}
\begin{tabu}{|p{9cm}|l|l|}
\hline
Sensorinformation&Format&frequency [$s^{-1}$]\\
\hline
App identifier for vendor & int64 & once per transfer\\
WIFI and network carrier IP addresses& int128 & once per transfer\\
battery level& int8 & 0.1\\
Position information: latitude, longitude, altitude, speed, course, vertical position accuracy, horizontal position accuracy, floor level information& float32[8] & 1\\
Heading information: heading.x, heading.y, heading.z, true heading, magnetic heading, heading accuracy& float16[6] & 1 \\
Accelerometer information: acceleration.x, acceleration.y, acceleration.z& float16[3] & 2 \\
Gyroscope information: rotationRate.x, rotationRate.y, rotationRate.z& float16[3] & 2 \\
altimeter information: relative altitude, pressure & float16[2] & 1 \\ 
timestamp & uint32 & once per transfer \\
Temperature [°C] & float16 & 1\\
\hline
\end{tabu}
\label{tab:smartphonesensor}
\end{table}

Wie in Tabelle \ref{tab:smartphonesensor} zu sehen ist, ist es besser, Trennlinien nur dort einzusetzen, wo logische Grenzen liegen.
%\include{./doc/111Schluss}
\clearpage
%\section{Abstract}
%\label{sec:Abstract}
%\chapter*{Abstract}
%\label{chap:Abstract}
\par\noindent
\textit{\textbf{Abstract}} Modern day railway system operations require automated train control mechanisms, e.g. European Train Control System ETCS, to maximize efficiency, which is often times limited by outdated infrastructure, as well as safety of operations. One way to achieve this is by lowering the required distance between two trains on the same track, which in turn demands a reliable method of predicting the braking distance at any given moment. 
\par
While determination of the necessary braking curves is feasible for a limited number of train formations, the large diversity of vehicles in freight operations poses an issue. One approach for a solution would be using Big Data, which would be able to process the required amounts of data to calculate reliable braking curves even for freight operations. 
\par
The problem here is that there is simply not enough data available since freight trains usually don't have the sensory equipment needed. To circumvent that obstacle, this work proposes generation of artificial data via white box modeling to be then used in further big data operations.
\chapter{Introduction}
\label{chap:Introduction}
\par\noindent
\textit{\textbf{Introduction}} This section describes the background and motivation of the research (Sect. \ref{sec:Background}), the problem to be addressed (Sect. \ref{sec:Problem}) and the proposed solution (Sect. \ref{sec:Solution}) 

\section{Background}
\label{sec:Background}
\par\noindent
In recent years, data science has become more and more prominent. Since large quantities of data have become omnipresent, big data processing is applicable in various fields of research and operations. This work focuses of the application of big data processing to railway system operations, more specifically the preconditions. Mainly, two areas show promise to benefit from big data usage, which is predictive maintenance as well as optimization of railway operations. As the name suggests, lots of data is required, think hundreds of terabyte for freight operations in Germany alone. As it stands though, unfortunately, this data source remains largely untapped today, owing to freight wagons typically lacking necessary sensory equipment. This is where this work comes into play.

\subsection{Railway Vehicle Operations}
\label{sec:RailwayVehicleOperations}
\par\noindent
Road and rail traffic are fundamentally different, mainly in two regards. First, the physical properties, which will be discussed in chapter \ref{chap:FundamentalsOfRailwayVehicleEngineering}, second, the actual modalities of operations, which will be topic of this section.
\par
In road traffic, there is many different vehicles, all operating independently from one another. Although there is ongoing experimentation to utilize autonomous vehicles to recreate train-like lorry configurations, this is not the norm. In contrast to that, the track guiding of wheels offered by rails enables the formation of trains possibly kilometers long, reducing labor cost, infrastructure usage and energy consumption. This, of course, calls for special safety measures to be taken, since higher speeds and payloads result in long braking distances. 

\begin{figure}[H]
	\centering
	\includegraphics[width=\linewidth]{./pic/abstaende}
	\caption{Spacing paradigms}
	\label{fig:train_spacing}
\end{figure}

\par\noindent
The above figure illustrates the three main principles of spacing between two trains on the same track. $s_{b,i}$ denoted the braking distance of train $i$, $t_{l,i}$ the length of train $i$, $l_{bl}$ the length of a track section, and S a safety margin. Fixed spatial distance, the most commonly used today, is very inefficient in terms of track utilization, but offers the most protection against accidents. In order to optimize efficiency, it is desirable to move towards relative braking distances, which requires the ability to very accurately predict braking capacity of trains.

\subsection{Automatic Train Protection and Cab Signaling}
\label{sec:ATPCS}
\par\noindent
Automatic train protection systems are designed to ensure safe operation in case of human or technical error or malpractice by the train operator. Generally, all trains operate on track sections which are free of other vehicles, reserved and locked. Think for example a section between two signals. These sections are also referred to as movement authority, and a train has to be capable of coming to a complete halt before the end of movement authority as to not violate a track section locked by another train. For this, the driver and or the onboard computer needs to be informed about the endpoint of the current movement authority as well as speed limits. The train protection system enforces application of brakes in case of violation of restrictions, for example if the train exceeds the speed limit or the train would otherwise be unable to stop in time before movement authority expires. Train protection systems may be categorized by means of information transfer. Spatially discrete acting systems use transmitters placed at strategic points along the track, for example signals, whereas continuous systems may transmit information at all times, either via track-sided wire loops or radio communication. These allow higher degrees of automation than discrete systems, as trains have access to the necessary real-time data.
\par
Cab signaling relays all relevant information to the train operator so they may act in the most optimal way. The European train control system, short ETCS, encompasses both of these.

\subsection{Braking Curves}
\label{sec:BrakingCurves}
\par\noindent
For the ETCS to be able to supervise train velocity, it needs to determine the vehicle's braking capacity for any given moment in time, using a mathematical model of the braking dynamics and of the track characteristics. This prediction is called a braking curve, which describe the probable braking process for a set of input parameters. As a simple example, refer to figure \ref{fig:brakingcurves}. Since braking performance cannot be predicted with absolute certainty, an $\epsilon$ parameter needs to be factored in to account for random behavior. Therefore, there is a number of discrete braking curves for any set of parameters, forming a probability distribution. The parameters may be classified in four categories:
\begin{itemize}
	\item Physical parameters, which is real time measurements from on-board equipment
	\item Fixed values, like driver reaction time
	\item Trackside data, like track gradient or signaling data
	\item Train data, mostly captured beforehand, relating to the rolling stock braking system itself
\end{itemize}
The fourth category, train data, is the key factor for this work, as will be explained below.

\begin{figure}[H]
	\centering
	\includegraphics[scale=0.2]{./pic/171026_Transrail_BrakingCurves-14}
	\caption{Braking curves}
	\label{fig:brakingcurves}
\end{figure}

\section{Problem}
\label{sec:Problem}

\par\noindent
As mentioned, one category of input parameters for the calculation of braking curves is data directly associated with the rolling stock, and in particular to the braking system itself. While this does not necessarily present itself as an issue in passenger trains, freight wagons, in contrast, are usually not electrified and therefore do not currently posses the sensory equipment that would be necessary to obtain such data in an adequate quantity and quality, especially in regards to \textit{big data processing}. Although it has been proposed to equip freight wagons accordingly (Refer to the concept \enquote{freight wagon 4.0} by Dr. Manfred Enning and Dr. Raphael Pfaff), it will be years, possibly decades, before enough rolling stock has been retrofitted as to make it possible to obtain the desired data. Since braking curves vary according to the rolling stock, this lack of data makes calculation thereof impossible.

\section{Solution}
\label{sec:Solution}

\par\noindent
Since the problem is lack of data from real life operations, this work proposes to circumvent this by generation of artificial, simulated data instead. The data set should replicate the actual distribution of braking behavior as close as possible. It is therefore necessary to first create a model encompassing the braking process of a freight train. This model will be discussed in depth in chapter \ref{chap:ModelingOfTrainOperations}. By using that model to run a large number of simulated braking processes, one may obtain data about the behavior of the braking system, which, albeit being artificial, should at least satisfy requirements for the calculation of rudimentary braking curves. A freely configurable simulation environment further allows for great flexibility in terms of input parameters, therefore enabling for covering a very large range of data, where computing power and time are the only limiting factors. The process of data generation and simulation will be discussed further in chapter \ref{chap:DataGeneration}.
\par
As real life operations would yield very high quantities of data, the simulation output must be stored in a data structure which is suitable for big data processing. This structure will also be discussed in chapter \ref{chap:DataGeneration}.
\chapter{Fundamentals of Railway Vehicle Engineering}
\label{chap:FundamentalsOfRailwayVehicleEngineering}
\par\noindent
\textit{\textbf{Introduction}} This section deals with the engineering backgrounds of the work, especially in regards to railway vehicle engineering and railway physics. 
For the creation of a braking model, it is necessary to have at least a basic understanding of the engineering and physics behind rail traffic. The most fundamental component to understand is, of course, the actual braking process.

\section{The train brake}
\label{sec:TrainBrake}
\par\noindent
There is a number of different construction methods for brakes, but each must address two problems: The means of transmission of commands and the actual form of brake force generation. The focus here will lie on the pneumatic brake, since it is still the most prevalent in cargo vehicles. From a macro perspective, the pneumatic brake has three main components: The train driver's brake valve, the brake pipe and one or more brake cylinders for each wagon or locomotive. The driver's valve and the brake pipe are primarily the means of command transmission, while the brake cylinder's job is the generation of brake force. 

\par
The braking system is an indirect one, which means the brake pipe needs to be vented to apply the brakes. Therefore, the brake pipe is kept at a running pressure of 5 bar. One advantage of this approach is that brakes are automatically applied in case of train separation. The downside, however, is that command transmission is limited to the speed of sound. In order to generate that pressure, the locomotive contains a compressor and an air reservoir. Initially, the brake pipe as well as all auxiliary air reservoirs are brought to operating pressure. If the train operator wishes to brake, he has to actuate the driver's brake valve, which in turn vents the brake pipe, usually as far as 3.5 bar (full braking), sometimes even lower. As rule of thumb, the more pressure is vented, the more braking force is applied. Each wagon brake has, in essence, an auxiliary air reservoir, a distributor valve and a braking cylinder. When the brake pipe is vented, the air in the auxiliary reservoir has a higher pressure. This difference in pressure triggers a switch of the distributor valve, and the air from the reservoir may go into the brake cylinder. This increase in the cylinder generates the actual braking force. To summarize, the amount of generated braking pressure is directly proportional to the amount of pressure vented.

\section{Influences on braking performance}
\label{sec:InfluencesOnBraking}
\par\noindent
Since the aim of this work is to monitor braking performance, it becomes necessary to understand which factors influence the braking process, and how. The braking process is, in its core, a system. A system may be defined as a set of objects, which are interconnected by relations. It is enclosed by its environment, which may or may not affect the system itself. Below is an illustration, also called box model:

\begin{align*}
	\displaystyle {\mathcal {R}}\ 
	\rightarrow \ 
	{\begin{array}{|c|}\hline \quad \\\quad {\mathcal {V}}_{i}\leftrightarrow {\mathcal {V}}_{n}\quad \\\quad \\\hline \end{array}}\ 
	\rightarrow
\end{align*}

\par\noindent
The box represents the actual system, ${\mathcal{V}}_{i}$ are the system variables, or objects, their relations represented by the $\leftrightarrow$. The ${\mathcal{R}}$ represents the system's environment and its influence on the system itself. Finally, the rightmost arrow $\rightarrow$ is the influence the system might have on the environment, though this is often disregarded as the focus lies on what happens inside the system. Creating a box model for the braking process makes for an easier identification of the influencing factors, and allows for their categorization into inherent, i.e. system variables, and external, i.e. system environment.
\par
From a modeling perspective, it is sensible to differentiate as follows: Everything related to the train itself is part of the system, and everything else is part of the system's environment. As per definition, it is compromised of a set of variables, a set of relations, and a set of constants. Let us make a formal definition of the model for the braking process system:

\begin{align*}
{\mathcal {M}} = \{ \{{\mathcal {V}}\}, \{{\mathcal{R}}\}, \{c\} \}
\end{align*}

\noindent
where $\{{\mathcal {V}}\}$ is the set of system variables, $\{{\mathcal {R}}\}$ is the set of relations (a) between elements of $\{{\mathcal {V}}\}$ and (b) between system and environment, and $\{c\}$ is the set of system constants. The system variables may be subsystems themselves. We shall take a closer look at $\{{\mathcal {V}}\}$ first.

\subsection{Inherent factors, model variables}
\label{sec:InherentFactors}
\par\noindent
Looking at ${\mathcal{M}}$, which is the system describing the braking process, $\{{\mathcal {V}}\}$ consists of all factors inherent to the train which have an impact on the braking, as well as the relevant quantifiers. The main ones are:
\begin{itemize}
	\item The train brake
	\item The wagon mass, denoted as $m$
	\item The train velocity, denoted as $v$
	\item The train composition, i.e. in which order the wagons are, for example the heaviest wagons being in front and the lighter ones at the back, denoted as $comp$
	\item The generated braking force, denoted as $F_{b}$
	\item The train's deceleration, denoted as $a$
	\item The train's braking distance, denoted as $d$
\end{itemize}
\noindent
It is noteworthy that the train brake is another system in itself, which is perfectly fine of course. Alas, it is therefore necessary to define another model.

\begin{align*}
{\mathcal {M}}_{B} = \{ \{{\mathcal {V}}_{B}\}, \{{\mathcal{R}}_{B}\}, \{c_{B}\} \}
\end{align*}

\noindent
Let ${\mathcal{M}}_{B}$ be the model describing the system train brake, with $\{{\mathcal {V}}_{B}\} = \{ {\mathcal {M}}_{bv}, {\mathcal {M}}_{bp}, {\mathcal {M}}_{bc} \}$, where ${\mathcal {M}}_{bv}$ is the model of the brake valve, ${\mathcal {M}}_{bp}$ the model of the brake pipe and ${\mathcal {M}}_{bc}$ the model of the brake cylinder, which in turn shall be defined as follows:

\begin{align*}
{\mathcal {M}}_{bv} = \{ \{{\mathcal {V}}_{bv}\}, \{{\mathcal{R}}_{bv}\}, \{c_{bv}\} \}
\end{align*}

\noindent
Let ${\mathcal {M}}_{bv}$ be the model describing the system brake valve, with ${\mathcal {V}}_{bv} = \{x\}$, ${\mathcal {R}}_{bv} = \emptyset$ and $c_{bv} = \emptyset$, where $x$ is the state of the brake valve, i.e. its opening percentage, ranging from 0 (fully closed) to 1 (fully open, full braking).

\begin{align*}
{\mathcal {M}}_{bp} = \{ \{{\mathcal {V}}_{bp}\}, \{{\mathcal{R}}_{bp}\}, \{c_{bp}\} \}
\end{align*}

\noindent
Let ${\mathcal {M}}_{bp}$ be the model describing the system brake pipe, with ${\mathcal {V}}_{bp} = \{ l_{bp}, p_{bp} \}$, ${\mathcal {R}}_{bp} = \emptyset$ and $c_{bp} = \{ v_{bp} \}$, where $l_{bp}$ is the physical length of the brake pipe, $p_{bp}$ is the pressure on the brake pipe and $v_{bp}$ is the propagation velocity of the pipe's medium. 

\begin{align*}
{\mathcal {M}}_{bc} = \{ \{{\mathcal {V}}_{bc}\}, \{{\mathcal{R}}_{bc}\}, \{c_{bc}\} \}
\end{align*}

\noindent
Let ${\mathcal {M}}_{bc}$ be the model describing the system brake cylinder, with ${\mathcal {V}}_{bc} = \{ p_{bc} \}$, ${\mathcal {R}}_{bc} = \emptyset$ and $c_{bc} = \{ t_{bc} \}$, where $p_{bc}$ is the cylinder's pressure and $t_{bc}$ is the cylinder's fill time.
\bigskip
\par\noindent
This wraps up the definition of ${\mathcal{V}}_{B}$. We may now get to the definition of ${\mathcal{R}}_{B}$, which is the relations (a) between the elements of ${\mathcal{V}}_{B}$ and (b) between ${\mathcal{M}}_{B}$ and its environment. Let us begin with (a):
\bigskip
\par\noindent
\textit{\textbf{Convention}} For readability purposes, $\propto$ shall from here on denote direct proportionality, and $\sim$ shall denote inverse proportionality.
\bigskip
\par
${\mathcal{M}}_{bv}$ relates to ${\mathcal{M}}_{bp}$ in the sense that the state of the brake valve has a direct influence on the pressure on the brake pipe. More specifically, $x \in {\mathcal{V}}_{bv}$ is inversely proportional to $p_{bp} \in {\mathcal{V}}_{bp}$. The higher the value of $x$, i.e. the opening percentage of the brake valve, the lower the value of $p_{bp}$, i.e. the pressure on the brake pipe. We can therefore denote $x \sim p_{bp}$, and subsequently define a relation $R_{bv,bp}: {\mathcal{M}}_{bv} \sim {\mathcal{M}}_{bp}$. 
\par
${\mathcal{M}}_{bp}$ relates to ${\mathcal{M}}_{bc}$ in the sense that the pressure on the brake pipe has a direct influence on the brake cylinder's pressure. More specifically, $p_{bp} \in {\mathcal{V}}_{bp}$ is inversely proportional to $p_{bc} \in {\mathcal{V}}_{bc}$. The lower the value of $p_{bp}$, i.e. the pressure on the brake pipe, the higher the value $p_{bc}$, i.e. the pressure in the brake cylinder. We can therefore denote $p_{bp} \sim p_{bc}$, and subsequently define a relation $R_{bp,bc}: {\mathcal{M}}_{bp} \sim {\mathcal{M}}_{bc}$.
\par
For simplicity, we can look at ${\mathcal{M}}_{B}$ as being isolated from its environment, i.e. the environment does not impact ${\mathcal{M}}_{B}$, and vice versa. Therefore, (b) is void. Consequently, we can define ${\mathcal{R}}_{B}$ as follows:

\begin{align*}
{\mathcal{R}}_{B} = \{ R_{bv,bp}, R_{bp,bc} \}
\end{align*}

\noindent
This finally leaves us with $c_{B}$. Since call constants are specific to the respective sub-models $\in {\mathcal{V}}_{B}$, ${\mathcal{M}}_{B}$ has no constants of its own, therefore $c_{B} = \emptyset$. This completes the definition of ${\mathcal{M}}_{B}$, and we can properly define the set of model variables of the model braking process:

\begin{align*}
\{ {\mathcal{V}} \} = \{ {\mathcal{M}}_{B},m,v,comp,F_{b},a,d \}
\end{align*}

\bigskip\noindent\TODO{
	Wie genau Bremsmodell definieren?
	Notizen: Bremse besteht aus, wie oben definiert:
	\begin{itemize}
		\item \sout{Brake valve - opening percentage $x$}
		\item \sout{Brake pipe - length $l_{bp}$, propagation velocity $v_{bp}$, pressure $p_{bp}$}
		\item \sout{Brake cylinder - fill time $t_{bc}$, pressure $p_{bc}$}
	\end{itemize}
	Relationen sind:
	\begin{enumerate}
		\item \sout{$x \sim p_{bp}$, intern}
		\item \sout{$p_{bp} \sim p_{bc}$, intern}
		\item $p_{bc} \sim F_{b}$, ausgangs-
		\item Wie zeitliche Auswirkungen darstellen?
	\end{enumerate}
}

\subsection{Relations}
\label{sec:Relations}
\par\noindent
Looking at our model ${\mathcal{M}}$, we have now defined its set of variables, ${\mathcal{V}}$. Next, it is necessary to define its set of relations, ${\mathcal{R}}$, which may be distinguished into internal and external relations.

\subsubsection{Relations between system variables}
\label{sec:RelationsSystemVariables}
\par\noindent
Obviously, the train brake ${\mathcal{M}}_{B}$ relates to $F_{B}$ as it is the actor responsible for generating the braking force. More specifically, $p_{bc} \in {\mathcal{V}}_{bc}$, which is the pressure in the brake cylinder, is directly proportionate to $F_{b}$, meaning the higher the pressure, the more force is applied. We can therefore denote $p_{bc} \propto F_{b}$, and subsequently define a relation $R_{{\mathcal{M}}_{B},F_{b}}: {\mathcal{M}}_{B} \propto F_{b}$.
\par
Furthermore, the wagon mass $m$ relates to $F_{B}$ as well

\bigskip\noindent\TODO{
	Wie interagieren die Systemvariablen untereinander? Relationen sind:
	\begin{itemize}
		\item \sout{${\mathcal{M}}_{B} \propto F_{b}$}
		\item $m \propto F_{b}$
		\item $v \propto d$
		\item $F_{B} \propto a$
		\item $a \propto d$
	\end{itemize}
}

\subsubsection{Relations between environment and system}
\label{sec:RelationsEnvironmentSystem}
\par\noindent
\TODO{\begin{itemize}
		\item Wheel Rail Friction
		\item Track gradient
\end{itemize}}

\section{What to monitor}
\label{sec:Monotoring}
\par\noindent
\TODO{
	Gehört das zu Modellvariablen??
	\begin{itemize}
		\item Brake force over time
		\item Brake pressure over time
		\item Deceleration
		\item Velocity
		\item Braking distance
	\end{itemize}
}
\chapter{Modeling of Train Operations}
\label{chap:ModelingOfTrainOperations}
\par\noindent
\emph{\textbf{Introduction}} As has been noted in \ref{sec:Solution}, it is necessary to model the braking process of freight trains. All modeling work has been performed with Matlab Simulink. 

\section{Initial Model}
\label{sec:InitialModel}
\par\noindent
The initial model to be expanded upon describes a single braking process. It's sole input, apart from some constants, is pressure over time, meaning a distinct value ranging between 5 and 3.5 bar for every timestamp. For visualization, please refer to \ref{fig:initmodel_siminput}. Let's take a look at the whole model first. 

\begin{figure}[H]
	\centering
	\includegraphics[width=\linewidth]{./pic/initmodel_whole}
	\caption{Initial Model}
	\label{fig:initmodel_whole}
\end{figure}

\par\noindent
Here we see a model of a freight train of fixed length, consisting of 40 wagons, which are, for better readability, further condensed to subsystems of five wagons each, so there are eight of these subsystems. They are interconnected via braking pipe, which is also the sole input to each system. Outputs are braking pressure and braking force. We will take a look at the actual wagon model next.

\begin{figure}[H]
	\centering
	\includegraphics[width=\linewidth]{./pic/initmodel_wagon}
	\caption{Initial Model - Wagon}
	\label{fig:initmodel_wagon}
\end{figure}

\par\noindent
Above is the initial wagon model. \NOTE{All 40 wagon models are identical here. This will be addressed in section \ref{sec:ModelExpansion}}. It consists of three main components.
\par
In the upper left corner is the input, which is the current pressure in the braking pipe. In the lower left corner, the propagation delay of the braking pipe is calculated. This is done by \TODO{}. Top center describes the calculation of the actual braking force, which is achieved by \TODO{}. Finally, the \TODO{Formulieren: Fahrzeugwiderstand}.

\section{Model Expansion}
\label{sec:ModelExpansion}

\par\noindent
This initial model is however not of sufficient detail. Where it merely describes one single braking process, we need to simulate a whole ride, with alternating phases of braking and accelerating. For that purpose, the simulation input has to be adjusted accordingly. Where previously it was only one braking process, using braking pressure as input was the obvious choice, whereas now the idea is to use a kind of track profile, which shall describe the maximum allowed velocity over time, of a notional track. For visualization, please refer to \ref{fig:expandedmodel_siminput}. The simulation then only needs to brake or accelerate depending on train velocity versus maximum velocity at the current time.

\par
Accordingly, the first expansion step is to create a mechanism to control the train so to speak. For this purpose, the system simply checks for each timestamp whether the current velocity of the train is greater than the maximum allowed velocity at the current time, according to simulation input. If this is the case, a braking pressure is applied to the pipe, scaling with the difference between $v_{max}$ and $v_{real}$, $v_{dif}$. This means the higher $v_{dif}$ is, the more braking pressure gets applied. This more or less covers the braking part of the system.

\par
The model however also needs a component for acceleration. To simplify things, the logic here is that if the train is not braking, it is accelerating, which actually works out pretty well. To accelerate, a traction force is applied, which also scales with $v_{dif}$, so the higher $v_{dif}$, the higher the applied traction force.

\begin{figure}[H]
	\centering
	\includegraphics[width=\linewidth]{./pic/expandedmodel_pressure}
	\caption{Expanded Model - Pressure Calculation}
	\label{fig:expandedmodel_pressure}
\end{figure}

\par\noindent
Depicted above is the system which determines braking pressure to apply. It calculates $v_{dif}$ by subtracting $v_{max}$ from $v_{real}$, which is then fed into a one-dimensional lookup table. The table is a sampled representation of a function with fixed breakpoints, mapping one function value to each breakpoint, like so

\begin{equation}
\label{eq:lookuptable}
H(n) =
\begin{cases}
0.1 & \text{if $n=1$} \\
0.7 & \text{if $n=15$} \\
0.8 & \text{if $n=20$} \\
\text{..}
\end{cases}
\end{equation}

\noindent
where $n$ is the breakpoints of $v_{dif}$. Since the pressure should only be applied if $v_{real}$ is greater than $v_{max}$, the ultimate result follows the logic of the following equation

\begin{equation}
\label{eq:brakingpressure}
P(n,t) = H(n) * (v_{real}(t) > v_{max}(t))
\end{equation}

\noindent
where $v_{real}(t)$ is train velocity over time, $v_{max}(t)$ is maximum velocity over time, and $v_{real}(t) > v_{max}(t)$ is either 1 or 0.

\begin{figure}[H]
	\centering
	\includegraphics[width=\linewidth]{./pic/expandedmodel_force}
	\caption{Expanded Model - Traction Force Calculation}
	\label{fig:expandedmodel_force}
\end{figure}

\par\noindent
The above system determines the traction force to apply. As has been discussed earlier, this works like a simple bang-bang controller. The design is very similar to the braking pressure system: $v_{dif}$ is again fed into a one-dimensional lookup table, which outputs different values for traction force accordingly. The higher $v_{dif}$ is, the higher the traction force to apply. It is then added to the current braking force, however only either traction or braking force is at any given time positive while the other is zero, which is achieved by the equations
 
\begin{equation}
\label{eq:tracforce}
f(n,t) = H(n) * (F_{B}(t) == 0) 
\end{equation}

\noindent
where $n$ is $v_{dif}$, $H(n)$ is the lookup table function (see equation \ref{eq:lookuptable}), $F_{B}(t)$ is the braking force over time, and 

\begin{equation}
\label{eq:brakeforce}
g(t) = F_{B}(t) * (F_{B}(t) \neq 0)
\end{equation}

\noindent
where $F_{B}(t)$ is the braking force over time, so we have 

\begin{equation}
\label{eq:force}
F(n,t) = f(n,t) + g(t) 
\end{equation} 

\noindent
where $F$ is the actual force over time, either braking or traction.

\par\noindent
$F$ is then used to calculate acceleration. According to Newton's Second Law,
\begin{equation}
\label{eq:newton}
F = m * a
\end{equation}
Accordingly, acceleration is
\begin{equation}
\label{eq:acceleration}
a = F(n,t) / m
\end{equation}
	
\noindent
where $m$ is the accumulated mass of all wagons and $F(n,t)$ relates to equation \ref{eq:force}. The acceleration is then used to calculate the velocity by integrating $a$ in relation to $v_{0}$, which is the initial velocity of the current braking or acceleration process \TODO{überprüfen..}. Integration of $v$ in turn allows calculation of the traveled distance.

\begin{figure}[H]
	\centering
	\includegraphics[width=\linewidth]{./pic/expandedmodel_wagon}
	\caption{Expanded Model - Wagon}
	\label{fig:expandedmodel_wagon}
\end{figure}

\par\noindent
The last subsystem is the actual wagon. We will take a look at the largely unchanged elements first. The sole input is still the braking pressure on the brake pipe. Simulation of the propagation delay has also remained the same as before. 

\par
One new addition is a pool of different wagons. Whereas before all 40 were distinguishable only by their position, they are now assigned with different parameters. To that end, a pool of 500 wagons has been randomly generated via python script, where each wagon has a unique ID, as well as randomly generated mass and braking efficiency. In actual simulation, up to 40 of these 500 are, currently by generation of random indices, selected and their properties used accordingly. It would also be possible to determine the wagon ids to be used beforehand, instead of choosing randomly.

\par
Another requirement was to make the number of wagons variable. In the initial model, the modeled train had a fixed number of 40 wagons, therefore also 40 wagon subsystems. Unfortunately, simulink offers no way to disable certain subsystems dynamically, but only by manually turning them off via model explorer, which would be unfeasible for such a large number of simulations. To circumvent this issue, output gets disabled for all unwanted wagons. For a simulation of a train of 20 wagons, the first 20 remain untouched, while the latter 20 produce no output and therefore also have no impact on the overall simulation. The turning off is achieved by simple switches; each wagon subsystem has a unique index from one to forty. If the index is greater than the specified number of wagons, all switches are turned to output zero.
	
\section{Further Expansion}
\label{sec:FurtherExpansion}
\section{Data Generation}
\label{sec:DataGeneration}

\subsection{Data Structure}
\label{sec:DataStructure}

\subsection{Analysis of generated Data}
\label{sec:AnalysisOfGeneratedData}
\section{Performance Analysis}
\label{sec:PerformanceAnalysis}
\chapter{Conclusion}
\label{chap:Conclusion}
\par\noindent
The aim of this work has been the creation of an artificial data set describing the braking processes of cargo trains in order to compensate for the lack of actual data from real life operations. To achieve this, the following working steps needed to be taken: First, it was necessary to develop an understanding of train operations in general, and the braking process of a train in particular. Here, the focus primarily lay on identifying the key parameters influencing the process. Once these had been worked out, it was possible to define a theoretical model ${\mathcal{M}}$ for the braking process. The model allowed us to formalize the inner workings of the system and how its different parts relate to each other, therefore serving as a basis for a practical implementation.
\par
Since a theoretical model can not be used for simulation and data generation purposes, the subsequent working step was the implementation of the model ${\mathcal{M}}$. Groundwork for this had already been laid in form of a rudimentary model of a cargo train of fixed length, breaking from an initial velocity to a full halt. As this model had been created with Mathwork's Simulink, it was decided to use this piece of software for this work, too. For the modeling work, Simulink proved to be an excellent choice, as it is intuitive to use and did not require a disproportionate amount of work to get used to. The expansion of the initial model, necessary to transform it into a complete implementation of the previously defined theoretical model, required quite a bit more time and effort, however, especially for someone not familiar with the field of railway vehicle engineering. Apart from the obvious challenges, a few problems appeared from time to time which were quite tricky to fix, mainly because Simulink does not offer much in terms of debugging.
\par
The implemented Simulink model could now finally be used to simulate a variety of braking processes, and record the behavior of the trains during these processes to create the artificial data set. In theory, this should have been the least laborious working step; considering the data set should be suitable for big data processing, finding a \textit{stable} way of generating large quantities of data proved to be a bit challenging. In the end, however, a satisfying result has been achieved, and more data sets may be generated, within an acceptable amount of time, as needed.
\par
Looking onward, this work could serve as basis for future efforts. On the one hand, the data sets may be used for data analyses, examples of application being predictive maintenance or further automation of train operations, leading towards autonomous trains and optimized utilization of infrastructure. On the other hand, the model of the braking process may be expanded further in order to achieve a more accurate representation of the real life process, and thus improve the quality of the generated data. Also, it might be feasible to optimize the performance of the data generation by means of runtime and code analysis.

\cite{Pfaff2017,Wende2003,Bosso2006,Pfaff2017a,Havryliuk2017,Cruceanu2012,Enning2017}


% Nachspann
% \nocite{Segmentation} % Quelle wird nicht im Text erwähnt -> Quellenverzeichnis
% \nocite{ImageAttack}
% Weitere quellen müssen in 'bib/quellen.bib' eingetragen werden
% !!! -> BibTex ausführen! Sonst tauchen die Quellen nicht im Verzeichnis auf.

% Quellenverzeichnis
\clearpage
% Quellenverzeichnis
\clearpage
\bibliographystyle{alpha}
%\bibliographystyle{apalike}
\bibliography{./bib/quellen}
\addcontentsline{toc}{chapter}{Quellenverzeichnis}
\addtocontents{toc}{\vspace{0.8cm}}

% Abkürzungsverzeichnis
%\clearpage
%\markright{Abkürzungsverzeichnis}
%\include{./doc/202abkuerzungsverzeichnis}
%\addtocontents{toc}{\vspace{0.8cm}}

% Abbildungsverzeichnis
\clearpage
\addcontentsline{toc}{chapter}{Abbildungsverzeichnis}
\listoffigures
\addtocontents{toc}{\vspace{0.8cm}}

% Tabellenverzeichnis
\clearpage
\addcontentsline{toc}{chapter}{Tabellenverzeichnis}
\listoftables
\addtocontents{toc}{\vspace{0.8cm}}

% Anhaenge
\addcontentsline{toc}{chapter}{Anhang}
\appendix
%\input{./app/Dateiname}
\chapter{Quellcode}
\begin{enumerate}
      \item Source 1
      \item Source 2
\end{enumerate}

% Anhänge im Ordner 'app' ablegen

%\includepdf[pages=1-4]{./app/Datenblatt1.pdf} % Datei mit 4 Seiten
%\includepdf[pages=1]{./app/Datenblatt2.pdf} % Datei mit einer Seite

\chapter{Data visualization}

%\includepdf{./app/initmodel_pressureinput.pdf}
\TODO{Visualisierungen einfügen}
\begin{figure}[H]
	\centering
	\includegraphics[width=\linewidth]{./pic/initmodel_subsys}
	\caption{Initial Model}
	\label{fig:initmodel_subsys}
\end{figure}

\begin{figure}[H]
	\centering
	\includegraphics[width=\linewidth]{./pic/initmodel_input}
	\caption{Initial Model}
	\label{fig:initmodel_input}
\end{figure}



\end{document}
