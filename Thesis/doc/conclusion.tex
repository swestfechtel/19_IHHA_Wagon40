\chapter{Conclusion}
\label{chap:Conclusion}
\par\noindent
The aim of this work has been the creation of an artificial data set describing the braking processes of cargo trains in order to compensate for the lack of actual data from real life operations. To achieve this, the following working steps needed to be taken: First, it was necessary to develop an understanding of train operations in general, and the braking process of a train in particular. Here, the focus primarily lay on identifying the key parameters influencing the process. Once these had been worked out, it was possible to define a theoretical model ${\mathcal{M}}$ for the braking process. The model allowed us to formalize the inner workings of the system and how its different parts relate to each other, therefore serving as a basis for a practical implementation.
\par
Since a theoretical model can not be used for simulation and data generation purposes, the subsequent working step was the implementation of the model ${\mathcal{M}}$. Groundwork for this had already been laid in form of a rudimentary model of a cargo train of fixed length, breaking from an initial velocity to a full halt. As this model had been created with Mathwork's Simulink, it was decided to use this piece of software for this work, too. For the modeling work, Simulink proved to be an excellent choice, as it is intuitive to use and did not require a disproportionate amount of work to get used to. The expansion of the initial model, necessary to transform it into a complete implementation of the previously defined theoretical model, required quite a bit more time and effort, however, especially for someone not familiar with the field of railway vehicle engineering. Apart from the obvious challenges, a few problems appeared from time to time which were quite tricky to fix, mainly because Simulink does not offer much in terms of debugging.
\par
The implemented Simulink model could now finally be used to simulate a variety of braking processes, and record the behavior of the trains during these processes to create the artificial data set. In theory, this should have been the least laborious working step; considering the data set should be suitable for big data processing, finding a \textit{stable} way of generating large quantities of data proved to be a bit challenging. In the end, however, a satisfying result has been achieved, and more data sets may be generated, within an acceptable amount of time, as needed.
\par
Looking onward, this work could serve as basis for future efforts. On the one hand, the data sets may be used for data analyses, examples of application being predictive maintenance or further automation of train operations, leading towards autonomous trains and optimized utilization of infrastructure. On the other hand, the model of the braking process may be expanded further in order to achieve a more accurate representation of the real life process, and thus improve the quality of the generated data. Also, it might be feasible to optimize the performance of the data generation by means of runtime and code analysis.