\chapter{Introduction}
\label{chap:Introduction}
\par\noindent
\emph{\textbf{Introduction}} This section describes the background and motivation of the research (Sect. \ref{sec:Background}), the problem to be addressed (Sect. \ref{sec:Problem}) and the proposed solution (Sect. \ref{sec:Solution}) 

\section{Background}
\label{sec:Background}
\par\noindent
In recent years, data science has become more and more prominent. Since large quantities of data have become omnipresent, big data processing is applicable in various fields of research and operations. This work focuses of the application of big data processing to railway system operations, more specifically the preconditions. Mainly, two areas show promise to benefit from big data usage, which is predictive maintenance as well as optimization of railway operations. As the name suggests, lots of data is required, think hundreds of terabyte for freight operations in Germany alone. As it stands though, unfortunately, this data source remains largely untapped today, owing to freight wagons typically lacking necessary sensory equipment. This is where this work comes into play.

\subsection{Railway Vehicle Operations}
\label{sec:RailwayVehicleOperations}
\par\noindent
Road and rail traffic are fundamentally different, mainly in two regards. First, the physical properties, which will be discussed in chapter \ref{chap:FundamentalsOfRailwayVehicleEngineering}, second, the actual modalities of operations, which will be topic of this section.
\par
In road traffic, there is many different vehicles, all operating independently from one another. Although there is ongoing experimentation to utilize autonomous vehicles to recreate train-like lorry configurations, this is not the norm. In contrast to that, the track guiding of wheels offered by rails enables the formation of trains possibly kilometers long, reducing labor cost, infrastructure usage and energy consumption. This, of course, calls for special safety measures to be taken, since higher speeds and payloads result in long braking distances. 
\par
\TODO{Abstandsparadigma}

\subsection{Train Protection Systems}
\label{sec:TrainProtectionSystems}
\par\noindent
Train Protection Systems are designed to ensure safe operation in case of human or technical error or malpractice by the train operator. 

\subsection{Braking Curves}
\label{sec:BrakingCurves}

\section{Problem}
\label{sec:Problem}

\par\noindent
As has been shown, to predict the braking behavior of trains, readings of wagons and locomotives are needed. Unfortunately, freight vehicles do not currently posses the sensory equipment that would be necessary to obtain such data in an adequate quantity and quality, especially in regards to \emph{big data processing}. Although it has been proposed to equip freight wagons accordingly \TODO{ref zu wagon4.0}, it will be years before enough rolling stock has been retrofitted as to make it possible to obtain the desired data.

\section{Solution}
\label{sec:Solution}

\par\noindent
This work proposes to circumvent the problem described above by creation of an artificial data set. The set must replicate the actual distribution of braking behavior as close as possible. It is therefore necessary to first create a model encompassing the braking process of a freight train. This model will be discussed in depth in chapter \ref{chap:ModelingOfTrainOperations}. It can then, once finished, be also used to generate the data set by simulation. This process will be discussed in chapter \ref{chap:DataGeneration}.
\par
As real life operations would yield very high quantities of data, the simulation output must be stored in a data structure which is suitable for big data processing. This structure will also be discussed in chapter \ref{chap:DataGeneration}.