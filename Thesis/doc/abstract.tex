\clearpage
%\section{Abstract}
%\label{sec:Abstract}

\par\noindent
\textit{\textbf{Abstract}} Modern day railway system operations require automated train control mechanisms, e.g. European Train Control System ETCS, to maximize efficiency, which is often times limited by outdated infrastructure, as well as safety of operations. One way to achieve this is by lowering the required distance between two trains on the same track, which in turn demands a reliable method of predicting the braking distance at any given moment. 
\par
While determination of the necessary braking curves is feasible for a limited number of train formations, the large diversity of vehicles in freight operations poses an issue. One approach for a solution would be using Big Data, which would be able to process the required amounts of data to calculate reliable braking curves even for freight operations. 
\par
The problem here is that there is simply not enough data available since freight trains usually don't have the sensory equipment needed. To circumvent that obstacle, this work proposes generation of artificial data via white box modeling to be then used in further big data operations.